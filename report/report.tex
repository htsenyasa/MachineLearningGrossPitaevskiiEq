\documentclass[a4paper,times,12pt]{article}
\usepackage{indentfirst}
\usepackage{amsmath}
\usepackage{graphicx}
\usepackage{float}
\usepackage{setspace}
\usepackage{svg}
\usepackage{subcaption}
%\usepackage{minted}
\onehalfspacing
\usepackage[top=2.5 cm, bottom=2.5 cm, left=4 cm, right=2.5 cm]{geometry}
\usepackage{hyperref}


\begin{document}

\begin{titlepage}
\begin{center}
\vspace*{1cm}
\underline{\textbf{\Large ISTANBUL TECHNICAL UNIVERSITY}} \\[10 pt]

\underline{\textbf{\large FACULTY OF SCIENCE AND LETTERS}} \\[15 pt]

\textbf{\large Graduation Project} \\
\vspace{1.8 cm}
\includegraphics[scale=1.2]{itu_logo.pdf} \\
\vspace{1.8 cm}
%\textbf{\large Machine Learning and Nonlinear Schr{\"o}dinger Equation} \\[5 pt]
\textbf{\large Machine Learning and Non-linear Schr{\"o}dinger Equation} \\[5 pt]
\textbf{H{\"u}seyin Talha \c{S}enya\c{s}a}\\
\vspace{1.5 cm}
\end{center}
\vfill
\noindent\textbf{{Department : Physics Engineering}}\\
    \textbf{Student ID \hspace{0.2 cm}: 090120132}\\
    \textbf{Advisor\hspace{1.1 cm}: Assoc. Prof. A. Levent Suba\c{s}{\i}}
\vspace{2 cm}

\center\textbf{FALL 2017}

\end{titlepage}


%\title{\textbf{TEST}}
%\date{}
%\maketitle
\setcounter{page}{1}
\pagenumbering{roman}

\section*{Summary}

We train an artificial neural network to estimate the ground state energy
of a one-dimensional Bose-Einstein condensate in harmonic trapping potential.
Such a system can be described by the solution of a non-linear Schr{\"o}dinger equation also called a Gross-Pitaevskii equation. We also use the method for the inverse problem of predicting the non-linearity parameter using the ground
state density profile for a given harmonic trapping potential.

\newpage
\tableofcontents

\newpage

\pagenumbering{arabic}
\section{Introduction and Motivation}
\label{sec:Intro}
\noindent 
Machine learning.\\
General usage area.\\
ML in physics and Phyiscs in ML.\\
ML\&SE article.\\
Ours difference.\\


\section{Gross Pitaevskii Equation}
\noindent   
General information about GPE \\
Why and how nonlinearty is introduced.\\
Physical and mathematical interpretation of interaction parameter. (phy: attractive, repulsive  math:dominance of the terms)\\
Stationary form.\\
Potential, kinetic and interaction energy expressions.\\
Reduction of dimension.\\
Analytic solution and approximation.

\subsection{Numeric Solution and Dataset Generation}
\noindent 
Potential types (with analytic forms etc.)\\ 
Scaling, $\alpha \beta$ case etc \\
Brief info about imaginary time evolution. (detailed in APPENDIX)\\
XMDS framerwork and other programs.\\
Potential generation.\\
Random potential generations with different method. (Reason)\\ 
Boundaries. (Table)\\
Convergence (detailed in APPENDIX).\\
Dataset generation. (Total number of examples etc)\\    

\subsubsection{Scaling}

$$ V(z) = \widetilde{V}(z)\gamma E_0 $$

$$z = \beta L\widetilde{z}$$

$$\mu =  \widetilde{\mu}\gamma E_0$$

$$ \frac{\hbar^2}{2m\gamma E_0} \frac{1}{\beta^2 L^2} = \alpha $$

$$ \psi = \widetilde{\psi}\sqrt{\frac{N}{\beta L}}  $$


\begin{equation}
    \label{eq:GPE_dimensionless}
    -\alpha\frac{d^2\widetilde{\psi}}{d\widetilde{z}^2} + \widetilde{V}(z)\widetilde{\psi} + \widetilde{g}|\widetilde{\psi}|^2 \widetilde{\psi} = \widetilde{\mu} \widetilde{\psi}
\end{equation}
    
\subsection{Dataset Features}
\subsubsection{Energy distribution}

\svgpath{path = {"../figs/dataresults/"}}
\begin{figure}[H]
    \centering
    \begin{subfigure}[t]{0.45\textwidth}
		%\centering
        \includesvg[width=\linewidth]{gaussian-total}
        \caption{g = 0}
		\label{fig:a}
    \end{subfigure}
    \caption{Total energy distributions of the generated solution for different interaction parameter values.}
\label{fig:energy_dist}
\end{figure}


\section{Machine Learning}
\subsection{Network architecture}
\noindent Architecture of the network.\\
    A general figure like in the ML\&SE article that describes the work done.\\
    Another figure about internals of the network such as number of layers, how interaction parameter is introduced to the network etc.\\
    Hyperparameters.\\
\subsection{Training}
\noindent    Detailed info about dataset (energy distribution etc).\\
    Indicate that if there is any method to increase the number of examples in low and high energy values.

\subsection{Results}

\section{Inverse Problem}

\clearpage
\section{Conclusion and Discussion}

\noindent Conclusion.\\
    Discussion.\\
    Effects of random potential generation method.\\
    Are there problems in low and high energies compared to the mean?\\
    Inverse problem.\\


\clearpage


\appendix
\section{APPENDIX A}



\end{document}