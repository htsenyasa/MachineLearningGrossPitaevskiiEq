\documentclass[a4paper,times,12pt]{article}
\usepackage{amsmath}
\usepackage{graphicx}
\usepackage{float}
\usepackage{setspace}
\onehalfspacing
\usepackage[top=2.5 cm, bottom=2.5 cm, left=4 cm, right=2.5 cm]{geometry}
\usepackage{hyperref}

\title{}%\textbf{Interaction Effects in Quantum Random Walk with atomic BEC}}

\date{}

\begin{document}


%\maketitle
\setcounter{page}{1}
\pagenumbering{roman}

\section*{Summary}
Lorem impus

\section{Gross Pitaevskii Equation}
\section{Problem Statements and Dataset Generation}
\section{Machine Learning for NLSE}

We implemented two different types of neural network and trained them for each interaction parameter separately.

We used Pytorch Framework to implement our neural networks. 


\subsection{Pytorch}


\subsection{Architecture}

As it is mentioned, there are two different architectures; feed forward (FNN) and convolutional neural network (CNN).  

FFN involves 128 input neurons as input layer, next layer is first hidden layer with 30 neurons, the second is same as first hidden layer, the next one is last hidden layer with 10 neurons and the last layer is output layer. Totally there are 5 layers in our FFN and we will denote as $FFN[128, 30, 30, 10, 1]$. 



\subsection{Hyperparameters}


\newpage

\end{document}