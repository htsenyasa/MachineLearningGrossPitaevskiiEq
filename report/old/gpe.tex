              
                % ****** Start of file apssamp.tex ******
%
%   This file is part of the APS files in the REVTeX 4.1 distribution.
%   Version 4.1r of REVTeX, August 2010
%
%   Copyright (c) 2009, 2010 The American Physical Society.
%
%   See the REVTeX 4 README file for restrictions and more information.
%
% TeX'ing this file requires that you have AMS-LaTeX 2.0 installed
% as well as the rest of the prerequisites for REVTeX 4.1
%
% See the REVTeX 4 README file
% It also requires running BibTeX. The commands are as follows:
%
%  1)  latex apssamp.tex
%  2)  bibtex apssamp
%  3)  latex apssamp.tex
%  4)  latex apssamp.tex
%
\documentclass[%
 reprint,
%superscriptaddress,
%groupedaddress,
%unsortedaddress,
%runinaddress,
%frontmatterverbose, 
%preprint,
%showpacs,preprintnumbers,
%nofootinbib,
%nobibnotes,
%bibnotes,
 amsmath,amssymb,
 aps,
%pra,
%prb,
%rmp,
%prstab,
%prstper,
%floatfix,
]{revtex4-1}

\usepackage{graphicx}% Include figure files
\usepackage{dcolumn}% Align table columns on decimal point
\usepackage{bm}% bold math
%\usepackage{hyperref}% add hypertext capabilities
%\usepackage[mathlines]{lineno}% Enable numbering of text and display math
%\linenumbers\relax % Commence numbering lines

%\usepackage[showframe,%Uncomment any one of the following lines to test 
%scale=0.7, marginratio={1:1, 2:3}, ignoreall,% default settings
%text={7in,10in},centering,
%margin=1.5in,
%total={6.5in,8.75in}, top=1.2in, left=0.9in, includefoot,
%height=10in,a5paper,hmargin={3cm,0.8in},
%]{geometry}

\begin{document}

\preprint{APS/123-QED}

\title{Machine Learning and Non Linear Schrodinger Equation}% Force line breaks with \\
%\thanks{A footnote to the article title}%

\author{Huseyin T. Senyasa}
%\altaffiliation[Also at ]{Physics Department, XYZ University.}%Lines break %automatically or can be forced with \\
%\author{Second Author}%
% \email{Second.Author@institution.edu}
%\affiliation{%
% Authors' institution and/or address\\
% This line break forced with \textbackslash\textbackslash
%}%
%
%\collaboration{MUSO Collaboration}%\noaffiliation
%
%\author{Charlie Author}
% \homepage{http://www.Second.institution.edu/~Charlie.Author}
%\affiliation{
% Second institution and/or address\\
% This line break forced% with \\
%}%
%\affiliation{
% Third institution, the second for Charlie Author
%}%
%\author{Delta Author}
%\affiliation{%
% Authors' institution and/or address\\
% This line break forced with \textbackslash\textbackslash
%}%
%
%\collaboration{CLEO Collaboration}%\noaffiliation

\date{\today}% It is always \today, today,
             %  but any date may be explicitly specified

\begin{abstract}
...
%\begin{description}
%\item[Usage]
%Secondary publications and information retrieval purposes.
%\item[PACS numbers]
%May be entered using the \verb+\pacs{#1}+ command.
%\item[Structure]
%You may use the \texttt{description} environment to structure your abstract;
%use the optional argument of the \verb+\item+ command to give the category of each item. 
%\end{description}
\end{abstract}

\pacs{Valid PACS appear here}% PACS, the Physics and Astronomy
                             % Classification Scheme.
%\keywords{Suggested keywords}%Use showkeys class option if keyword
                              %display desired
\maketitle

%\tableofcontents

\section{\label{sec:level1}Gross-Pitaevskii Equation}

Since we are going to try to guess ground state energy for a given potential,
we should first able to generate train data. To do that, we require analytic
and numerical solutions for energy values. 

Gross-Pitaevskii Equation is given as;

$$ i\hslash \frac {\partial \psi}{\partial t} = \frac {-\hslash}{2m} \nabla^2
\psi + V(r)\psi + g|\psi|^2\ \psi  $$

The coefficient of the non-linear term determines the interaction type. 
If there is no interaction $g = 0$ and equation reduces the Schrodinger Equation. For repulsive interactions
$g > 0$, and $g < 0$ for attractive. \\


%de Broglie wavelength of the particles are nearly
%equal to the distance between particles and interactions in condensate are weak. These two conditions make us able to treat the condensation as
%a huge wave.\\
%Three body interactions and higher are negligible.
%
%The physical interpretation of the nonlinear term is that, at a given
%point in space, there is an energy contribution arising from the mean-field
%interactions of all the atoms in the immediate vicinity.


\textbf{Time Independent GPE}\\
If the given potential is independent of time, GPE is separable and there are
stationary solutions. \\


\textbf{Stationary Solutions}

When the potential is given, characteristic of the system is determined by
non-linear term's coefficient, g as shown below. Symmetric harmonic potential is,
$$ V(\textbf{r}) = \frac{1}{2}m\omega^2 r^2 $$

It is convenient to define characteristic length and interaction parameter (IP) to describe strength of the interactions which are given by respectively,

$$ l_r = \sqrt{\frac{\hslash}{m \omega_r}}$$ 
$$  \frac{N a_s}{l_r} $$ 

where $a_s$ is s wave scattering length.

if $ IP \gg 1 $, it corresponds to strong repulsive interaction, $ IP < 1 $, weak interactions.  

In no interaction case which means $g = 0$, GPE reduces to SE. This leads us to
obtain well known harmonic potential solutions. Energy is given by,

$$ E = \frac{3}{2}N \hslash \omega_r$$

In strong repulsive interaction, there is no analytic solution. We can obtain 
energy values by solving the GPE numerically or by approximation which is known as Thomas-Fermi approximation. Approximation requires negligence of the 
$\nabla^2\psi$ term. When the equation is solved and normalized condition is satisfied, energy is $$ E = \frac{5}{7} \mu N $$ and $$ \mu = \frac{\hslash \omega_r}{2} (\frac{15Na_s}{l_r})^\frac{2}{5} $$

\textbf{Reduction of Dimension}

The shape of the condensate can be controlled by applied potential. When harmonic potential is symmetric, shape of the condensate is spherical but if we
increase the $\omega_z$ component of the potential for fixed $\omega_y, \omega_x$, we can reshape it into pancake like shaped. In this case, condensate can be described by z component and system can be represented by 1D GPE which is given as,

$$ \mu_{1D}\psi_z = \frac {-\hslash}{2m} \frac{d^2\psi}{dz^2}
+ V(z)\psi_z + g_{1D}|\psi_z|^2\ \psi_z $$



(ground state density is directly proportional to s wave scattering length,
since g is function of s wave scattering length.)

(It also shows that for $g < 0$, system is not stable since density cannot be negative.)


\end{document}
%
% ****** End of file apssamp.tex ******
              
            